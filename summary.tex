\documentclass[12pt,titlepage,a4paper]{report}
\usepackage[utf8]{inputenc}
\usepackage[T1]{fontenc}
\usepackage{lmodern}
\usepackage{amsmath}
\usepackage{amsfonts}
\usepackage{amssymb}
\usepackage{graphicx}
\usepackage{wrapfig}
\usepackage{hyperref}
\usepackage{fullpage}
\usepackage[table]{xcolor}
\usepackage[english]{babel}

\renewcommand{\familydefault}{\sfdefault}


\author{Stijn Caerts}
\title{Development of Secure Software\\\small{Summary}}
\begin{document}
	\maketitle
	\tableofcontents
	\newpage
	
	\chapter{Introduction}
	All security incidents are consequences of vulnerabilities in the underlying systems (web servers, operating systems, applications).
	
	\section{Key-concepts}
	\subsection{Security goals or policy}
	\begin{itemize}
		\item \textbf{desirable properties} one wishes to maintain
		\item can be classified as Confidentiality, Integrity or Availability goals (CIA) of identified assets
		\begin{itemize}
			\item assets: information, services, infrastructure, ...
		\end{itemize}
	\end{itemize}

	\subsection{Adversary model}
	\begin{itemize}
		\item capabilities and resources of the \textbf{intelligent adversary} are made explicit
		\begin{itemize}
			\item bounded in some way, otherwise achieving security goal may be infeasible
			\item \emph{eg. the adversary cannot factor the product of two large primes}
		\end{itemize}
		\item \textbf{threat-driven} vs \textbf{goal-driven security}
		\begin{itemize}
			\item \textbf{threat-driven}: start by identifying potential threats against the system and come up with countermeasures
			\item \textbf{goal-driven}: start by eliciting security goals and come up with security mechanisms to guarantee them
			\item threats threaten specific assets
			\begin{itemize}
				\item \emph{eg. Spoofing, Tampering, Repudiation, Information disclosure, Denial-of-Service, Elevation of privileges (STRIDE)}
			\end{itemize}
		\end{itemize}
	\end{itemize}

	\subsection{Security argument}
	\begin{itemize}
		\item rigorous argument that under a given adversary model:
		\begin{itemize}
			\item a countermeasure counters the relevant threat, or
			\item a security mechanism achieves the relevant security goal
		\end{itemize}
	\end{itemize}

	\subsection{Vulnerability}
	\begin{itemize}
		\item aspect of the system that allows the adversary to break a security goal
		\item  can enter the system:
		\begin{itemize}
			\samepage
			\item early in the development life cycle
			\begin{itemize}
				\item failure to identify relevant security goals or adversaries
			\end{itemize}
			\item during construction of the system
			\begin{itemize}
				\item bugs in security mechanism
				\item incorrect security arguments: relying on abstractions that are not maintained in the presence of an intelligent adversary
			\end{itemize}
			\item during operation of the system
			\begin{itemize}
				\item bugs in the configuration of a security mechanism
			\end{itemize}
		\end{itemize}
	\end{itemize}

	\subsection{Countermeasures}
	\begin{itemize}
		\item types:
		\begin{itemize}
			\item Preventive: avoid vulnerability
			\item Detective: detect vulnerability exploitation
			\item Reactive: handle incidents
		\end{itemize}
		\item can be taken by various stakeholders
		\begin{itemize}
			\item Software Engineers
			\begin{itemize}
				\item early phases: security requirements engineering, threat analysis
				\item for threats discovered during RE $\rightarrow$ security technologies:
				\begin{itemize}
					\item cryptography, authentication mechanisms, access control, ...
				\end{itemize}
				\item for vulnerabilities during construction:
				\begin{itemize}
					\item secure programming, safe languages, static analysis, ...
				\end{itemize}
				\item for vulnerabilities during operation:
				\begin{itemize}
					\item documentation, operational procedures, secure defaults, ...
				\end{itemize}
			\end{itemize}
			\item Administrator
			\begin{itemize}
				\item Preventive:
				\begin{itemize}
					\item deployment of additional protection: Firewalls, VPN's, ...
					\item patching weakness where possible: security updates
				\end{itemize}
				\item Detective:
				\begin{itemize}
					\item Intrusion Detection or Fraud Detection software
					\item Virus scanning
				\end{itemize}
				\item Security solutions should be managed, supporting reactive countermeasures
			 \end{itemize}
		\end{itemize}
	\end{itemize}

	\section{Vulnerabilities in practice}
	\begin{itemize}
		\item "Securing" software = reducing the number of vulnerabilities in software, giving preference to those that contribute most to risk
		\item Important to know what vulnerabilities matter most in practice
		\begin{itemize}
			\item Researchers have been studying vulnerabilities (and their exploitations) for decades
		\end{itemize}
		\item Most vulnerabilities have to do with input/output validation or defensive programming
		\item \textbf{Software security} is strongly related to \textbf{software quality}
	\end{itemize}


	\chapter{Low-level Software Security}
	\section{Introduction}
	\subsection{Understanding execution of C programs}
	\begin{itemize}
		\item C code is compiled to machine code
		\item each function can be compiled separately
		\item control flow tracked by \emph{call-stack}
		\item variable location:
		\begin{itemize}
			\item local variables: on the call-stack
			\item global variables: statically
			\item using a memory management library for dynamically allocated storage (\texttt{malloc}/\texttt{new})
		\end{itemize}
	\end{itemize}
	\begin{table}[h!]
		\centering
		\begin{tabular}{| l | c}
			\cline{1-1}
			Arguments/Environment & High addresses \\ \cline{1-1}
			Stack & Stack grows down \\ \cline{1-1}
			\cellcolor{gray}Unused and Mapped Memory & \\ \cline{1-1}
			Heap (dynamic data) & Heap grows up \\ \cline{1-1}
			Static Data & \\ \cline{1-1}
			Program code & Low addresses \\ \cline{1-1}
		\end{tabular}
		\caption{Process memory layout}
	\end{table}

	\subsubsection{The call-stack}
	\begin{itemize}
		\item activation record
		\begin{itemize}
			\item arguments
			\item return address
			\item previous stack pointer
			\item automatically allocated local variables
		\end{itemize}
	\end{itemize}
	
	
	\subsection{Memory safety vulnerabilities}
	\begin{itemize}
		\item relevant for \textit{unsafe} languages
		\begin{itemize}
			\item languages that do not check whether programs access memory in a correct way
		\end{itemize}
	\end{itemize}
	
	\subsubsection{Types}
	\begin{itemize}
		\item Spatial safety errors
		\begin{itemize}
			\item index an array out-of-bounds
			\item invalid pointer arithmetic
		\end{itemize}
		\item Temporal safety errors
		\begin{itemize}
			\item use after free
			\item double free
		\end{itemize}
		\item Accessing uninitialized memory
		\item Unsafe \texttt{libc} API functions
		\begin{itemize}
			\item eg. \texttt{printf()}: format string vulnerabilities
		\end{itemize}
	\end{itemize}

	\subsubsection{Exploiting}
	\begin{itemize}
		\item C programs don't detect bugs at run-time
		\begin{itemize}
			\item behaviour of a buggy program is \emph{undefined}
			\item depends on compiler, OS, processor architecture, ...
			\item use knowledge of these lower layers to exploit the program
		\end{itemize}
	\end{itemize}

	\section{Attacks}
	\subsection{Stack-based buffer overflow}
	\begin{itemize}
		\item The stack is a memory area used at run-time to track function calls and returns
		\begin{itemize}
			\item per call: activation record containing return address, automatically allocated local variables, ...
		\end{itemize}
		\item by overflowing a local buffer variable, interesting memory locations can by overwritten
		\begin{itemize}
			\item simplest attack is to overwrite the return address so that it points to attacker-chosen code (\emph{shell code})
		\end{itemize}
		\item lots of details to get it right
		\begin{itemize}
			\item no nulls in (character-)strings: {\texttt{strcpy()} is terminated by null byte (\texttt{'\textbackslash0'})}
			\item filling in the correct return address:
			\begin{itemize}
				\item fake return address must be precisely positioned
				\item attacker might not know the address of his own string
			\end{itemize}
			\item other overwritten data must not be used before return from function
		\end{itemize}
	\end{itemize}

	\subsection{Heap-based buffer overflows}
	\begin{itemize}
		\item buffer on the heap that has overflow vulnerability
		\begin{itemize}
			\item no return address nearby, therefore overwrite other code pointers
		\end{itemize}
	\end{itemize}
	
	\subsubsection{Overwriting a function pointer}
	\begin{itemize}
		\item Overflow the buffer and overwrite a function pointer
		\begin{itemize}
			\item point back to malicious code placed in the buffer
			\item shell code gets executed when function would be called
		\end{itemize}
	\end{itemize}

	\begin{figure}[h]
		\centering
		\includegraphics*[scale=0.75]{assets/img/HeapBufferOverflowFunctionPointer.png}
		\caption{\label{img:heapBufferOverflowFuncionPointer}Overflow the buffer and overwrite the function pointer}
	\end{figure}
	
	\subsubsection{Overwriting heap meta data}
	
	
	\chapter{Web Security Fundamentals (MOOC)}
	\section{Is security an illusion?}
	\subsection{The web security landscape}
	\begin{itemize}
		\item Every website is valuable, even when no user data is stored. The websites' resources (storage, processing power, ...) are interesting for hackers.
		\item Security is often seen as an obstruction to functionality and productivity. It is often ignored until the very last moment.
		\begin{itemize}
			\item Penetration tests alone are not enough (may not find all threats, fundamental problem may require an entire redesign of the application, ...).
			\item Every developer should be aware about security and secure coding guidelines.
		\end{itemize}
	\end{itemize}

	\subsection{The security model of the web}
	\begin{itemize}
		\item URL: \texttt{[scheme]://[host]:[port]/[path]?[query]\#[fragment]}
		\begin{itemize}
			\item Fragment part is never sent to the server (client-side only).
		\end{itemize}
		\item Origin: scheme, host and port
		\item Same-Origin Policy (SOP)
		\begin{itemize}
			\item Contexts from the same origin can freely interact with each other, contexts from different origins are isolated.
			\item Browsing context is protected against undesired access.
		\end{itemize}
		
		\item Cookies
		\begin{itemize}
			\item belong to domains, not origins!
			\item key-value pair, used to track session information
			\item set by server, sent by browser
			\begin{itemize}
				\item stored in cookie jar (browser)
				\item for every outgoing request, the browser consults the cookie jar for that domain and automatically attaches them
				\item exchanged through HTTP or accessed via JavaScript
			\end{itemize}
			\item \texttt{Domain} attribute (determines scope)
			\begin{itemize}
				\item send cookie to all sub-domains of registered domain
			\end{itemize}
			\item \texttt{Path} attribute (determines scope)
			\begin{itemize}
				\item only attach cookie to requests to a resource within the path
			\end{itemize}
		\end{itemize}
		
		\item Client-centric security
		\begin{itemize}
			\item browser as an application platform
			\item need for extra security policies, under control of the server, but enforced by the browser
		\end{itemize}
	\end{itemize}

	\section{Securing the communication channel}
	
	
\end{document}